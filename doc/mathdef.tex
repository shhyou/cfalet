\usepackage{amsthm}
\usepackage{amsmath}
\usepackage{amsfonts}
\usepackage{amssymb}

\theoremstyle{definition}
\newtheorem{defn}[subsection]{Definition}
\theoremstyle{plain}
\newtheorem{prop}[subsection]{Proposition}
\newtheorem{lem}[subsection]{Lemma}
\newtheorem{thm}[subsection]{Theorem}
\newtheorem{cor}[subsection]{Corollary}
\theoremstyle{remark}
\newtheorem{ex}[subsection]{Example}
\newtheorem{rem}[subsection]{Remark}

\DeclareMathOperator{\dom}{dom}

\newcommand{\N}{\mathbb N}
\newcommand{\Z}{\mathbb Z}
\newcommand{\Q}{\mathbb Q}
\newcommand{\R}{\mathbb R}
\newcommand{\C}{\mathbb C}

\newcommand{\F}{\mathbb F}

\newcommand{\pow}{\mathcal P}

\newcommand{\set}[1]{\left\{#1\right\}}
\newcommand{\condset}[2]{\left\{\left. #1 \,\right|\, #2 \right\}}
\newcommand{\condsetr}[2]{\left\{ #1 \,\left|\, #2 \right.\right\}}
\newcommand{\blank}{\hphantom{....}}
\newenvironment{eq}{\[\begin{aligned}}{\end{aligned}\]}

% notations for algebra
\newcommand{\generate}[1]{\langle #1 \rangle}
\newcommand{\V}[1]{\mathbf{#1}}
\newcommand{\Mat}[1]{\begin{pmatrix}#1\end{pmatrix}}